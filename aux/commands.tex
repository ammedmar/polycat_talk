%!TEX root = ../polycat.tex

\newcommand{\colorit}[1]{\textcolor{pblue}{#1}}
\newcommand{\Bord}{\mathrm{Bord}}
\newcommand{\Vect}{\mathrm{Vect}}
\newcommand{\CC}{\mathbb{C}}

% From the paper

\DeclareMathOperator{\cov}{cov}

\usepackage{todo}
\def\todoname{Comments, remarks and to-do's}
\usepackage{csquotes}
\usetikzlibrary{arrows}
\usepackage{xargs}%Allows to define newcommands with several optional arguments with \newcommandx
\usepackage{comment}

\usepackage{tikz,tikz-cd,tikz-3dplot}
\usetikzlibrary{positioning,calc}

\renewcommand{\corollary}{\noindent\textit{Corollary}.\ }
\renewcommand{\theorem}{\noindent\textit{Theorem}.\ }
\renewcommand{\lemma}{\noindent\textit{Lemma}.\ }
\renewcommand{\proposition}{\noindent\textit{Proposition}.\ }


\newcommand{\faces}{\Gamma}
\DeclareMathOperator{\Span}{Span}
\DeclareMathOperator{\normal}{n}
\newcommand{\CP}{C}
\newcommand{\NC}{\cN}
\DeclareMathOperator{\Mor}{Mor}
\renewcommand{\bd}{\partial}
\newcommand{\oriental}{\cO}
\renewcommand{\face}{\operatorname{Face}}
\DeclareMathOperator{\cone}{Cone}
\DeclareMathOperator{\aug}{aug}
\newcommand{\wCat}{\omega\Cat}
\newcommand{\wDiag}{\omega\mathsf{Diag}}
\newcommand{\cells}{\nu}

%polytopes and polyhedra
\DeclareMathOperator{\conv}{conv} %convex hull

% source and target
\DeclareMathOperator{\so}{s} % source set
\DeclareMathOperator{\ta}{t} % target set

% alphabetically:
\renewcommand{\anibal}[1]{\todo[An.\!]{#1}}
\newcommand{\arnau}[1]{\todo[Ar.\!]{#1}}
\newcommand{\guillaume}[1]{\todo[G.\!]{#1}}

\renewcommand{\chains}{\gchains}

\newcommand{\sprod}[2]{\left\langle  #1 \, , \, #2  \right\rangle} % dot product

% \newcommand{\defn}[1]{\textsl{\darkblue #1}} % emphasis of a definition
\newcommand{\defn}[1]{\emph{#1}} % emphasis of a definition

\newcommand\restr[2]{{% we make the whole thing an ordinary symbol
		\left.\kern-\nulldelimiterspace % automatically resize the bar with \right
		#1 % the function
		\vphantom{\big|} % pretend it's a little taller at normal size
		\right|_{#2} % this is the delimiter
}}%function restriction


\newcommand{\bigslant}[2]{{\raisebox{.3em}{$#1$} \Big/ \raisebox{-.3em}{$#2$}}} % for big quotients


\DeclareMathOperator{\GL}{GL}
\newcommandx{\GLn}[1][1={\R^d}]{\GL(#1)} %General Linear group

\DeclareMathOperator{\Aff}{Aff}
\DeclareMathOperator{\Lin}{Lin}
\newcommandx{\Affn}[1][1={\R^d}]{\Aff(#1)} %Affine General Linear group

\newcommandx{\frames}[1][1=P]{\mathcal{F}_{#1}} %P-admissible frames

\newcommandx{\ms}[1][1=B]{\mathcal{M}({#1})} %moduli space of a frame

\newcommandx{\rs}[1][1=A]{\mathcal{R}({#1})} %realization space oriented matroid
\newcommandx{\rsinf}[1][1={\chi}]{\mathcal{R}_\infty({#1})} %realization space oriented matroid with a point at infinity

\newcommandx{\chiro}[1][1=A]{\chi^{#1}} %chirotope
\newcommandx{\cyclift}[1][1=A]{{#1}^{\uparrow}} %cyclic lift


\newcommand{\ball}[2]{B_{#2}(#1)} %Euclidean ball

%arrow

\newcommand{\thickmidarrow}{
	\begin{tikzpicture}

		\draw[ultra thick] (-2,1) -- (0,0) ;
		\draw[ultra thick] (-2,-1) -- (0,0) ;
\end{tikzpicture}}
