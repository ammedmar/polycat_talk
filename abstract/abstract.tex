\documentclass{article}
\usepackage{amssymb}

\begin{document}
	\pagestyle{empty}
	\begin{center}
		\vspace*{5pt}
		\Large\textsc{Framed Polytopes and Higher Structures} \\
		\vspace*{10pt}
		\normalsize Anibal M. Medina-Mardones \\
	\end{center}
	\textbf{Abstract.}
	A framed polytope is the convex closure of a finite set of points in $\mathbb{R}^n$ together with an ordered linear basis.
	An $n$-category is a category that is enriched in the category of $(n-1)$-categories. Although these concepts may initially appear to be distant peaks in the mathematical landscape, there exists a trail connecting them, blazed in the 90's by Kapranov and Voevodsky. We will traverse this path, widening and improving it as we address some of their conjectures along the way.
	
	Additionally, using a special embedding of the combinatorial simplex, we will connect this trail to the one ascending Mount Steenrod. This connection will enable us to express the combinatorics of cup-$i$ products in convex geometric terms, dual to those introduced earlier in the talk for defining the nerve of higher categories.
\end{document}